\documentclass[journal,12pt,twocolumn]{IEEEtran}
%
\usepackage{setspace}
\usepackage{gensymb}
\usepackage{xcolor}
\usepackage{caption}
%\usepackage{subcaption}
%\doublespacing
\singlespacing
\usepackage{polynom}
%\usepackage{graphicx}
%\usepackage{amssymb}
%\usepackage{relsize}
\usepackage[cmex10]{amsmath}
\usepackage{mathtools}
%\usepackage{amsthm}
%\interdisplaylinepenalty=2500
%\savesymbol{iint}
%\usepackage{txfonts}
%\restoresymbol{TXF}{iint}
%\usepackage{wasysym}
\usepackage{hyperref}
\usepackage{amsthm}
\usepackage{mathrsfs}
\usepackage{txfonts}
\usepackage{stfloats}
\usepackage{cite}
\usepackage{cases}
\usepackage{subfig}
%\usepackage{xtab}
\usepackage{longtable}
\usepackage{multirow}
%\usepackage{algorithm}
%\usepackage{algpseudocode}
%\usepackage{enumerate}
\usepackage{enumitem}
\usepackage{mathtools}
%\usepackage{iithtlc}
%\usepackage[framemethod=tikz]{mdframed}
\usepackage{listings}


%\usepackage{stmaryrd}


%\usepackage{wasysym}
%\newcounter{MYtempeqncnt}
\DeclareMathOperator*{\Res}{Res}
%\renewcommand{\baselinestretch}{2}

%\renewcommand{\labelenumi}{\textbf{\theenumi}}
%\renewcommand{\theenumi}{P.\arabic{enumi}}

% correct bad hyphenation here
\hyphenation{op-tical net-works semi-conduc-tor}

\lstset{
language=Python,
frame=single, 
breaklines=true,
columns=fullflexible
}



\begin{document}
%

\theoremstyle{definition}
\newtheorem{theorem}{Theorem}[section]
\newtheorem{problem}{Problem}
\newtheorem{proposition}{Proposition}[section]
\newtheorem{lemma}{Lemma}[section]
\newtheorem{corollary}[theorem]{Corollary}
\newtheorem{example}{Example}[section]
\newtheorem{definition}{Definition}[section]
%\newtheorem{algorithm}{Algorithm}[section]
%\newtheorem{cor}{Corollary}
\newcommand{\BEQA}{\begin{eqnarray}}
\newcommand{\EEQA}{\end{eqnarray}}
\newcommand{\define}{\stackrel{\triangle}{=}}

\bibliographystyle{IEEEtran}
%\bibliographystyle{ieeetr}

\providecommand{\nCr}[2]{\,^{#1}C_{#2}} % nCr
\providecommand{\nPr}[2]{\,^{#1}P_{#2}} % nPr
\providecommand{\mbf}{\mathbf}
\providecommand{\pr}[1]{\ensuremath{\Pr\left(#1\right)}}
\providecommand{\qfunc}[1]{\ensuremath{Q\left(#1\right)}}
\providecommand{\sbrak}[1]{\ensuremath{{}\left[#1\right]}}
\providecommand{\lsbrak}[1]{\ensuremath{{}\left[#1\right.}}
\providecommand{\rsbrak}[1]{\ensuremath{{}\left.#1\right]}}
\providecommand{\brak}[1]{\ensuremath{\left(#1\right)}}
\providecommand{\lbrak}[1]{\ensuremath{\left(#1\right.}}
\providecommand{\rbrak}[1]{\ensuremath{\left.#1\right)}}
\providecommand{\cbrak}[1]{\ensuremath{\left\{#1\right\}}}
\providecommand{\lcbrak}[1]{\ensuremath{\left\{#1\right.}}
\providecommand{\rcbrak}[1]{\ensuremath{\left.#1\right\}}}
\theoremstyle{remark}
\newcommand\Mydiv[2]{%
$\strut#1$\kern.25em\smash{\raise.3ex\hbox{$\big)$}}$\mkern-8mu
        \overline{\enspace\strut#2}$}
\newtheorem{rem}{Remark}
\newcommand{\sgn}{\mathop{\mathrm{sgn}}}
\providecommand{\abs}[1]{\left\vert#1\right\vert}
\providecommand{\res}[1]{\Res\displaylimits_{#1}} 
\providecommand{\norm}[1]{\lVert#1\rVert}
\providecommand{\mtx}[1]{\mathbf{#1}}
\providecommand{\mean}[1]{E\left[ #1 \right]}
\providecommand{\fourier}{\overset{\mathcal{F}}{ \rightleftharpoons}}
\providecommand{\ztrans}{\overset{\mathcal{Z}}{ \rightleftharpoons}}
\newcommand{\myvec}[1]{\ensuremath{\begin{pmatrix}#1\end{pmatrix}}}
\newcommand{\mydet}[1]{\ensuremath{\begin{vmatrix}#1\end{vmatrix}}}
%\providecommand{\hilbert}{\overset{\mathcal{H}}{ \rightleftharpoons}}
\providecommand{\system}{\overset{\mathcal{H}}{ \longleftrightarrow}}
	%\newcommand{\solution}[2]{\textbf{Solution:}{#1}}
\newcommand{\solution}{\noindent \textbf{Solution: }}
\providecommand{\dec}[2]{\ensuremath{\overset{#1}{\underset{#2}{\gtrless}}}}

%\numberwithin{equation}{subsection}
%\numberwithin{problem}{subsection}
%\numberwithin{definition}{subsection}
\makeatletter
\@addtoreset{figure}{problem}
\makeatother

\let\StandardTheFigure\thefigure
%\renewcommand{\thefigure}{\theproblem.\arabic{figure}}
\renewcommand{\thefigure}{\theproblem}


%\numberwithin{figure}{subsection}

\def\putbox#1#2#3{\makebox[0in][l]{\makebox[#1][l]{}\raisebox{\baselineskip}[0in][0in]{\raisebox{#2}[0in][0in]{#3}}}}
     \def\rightbox#1{\makebox[0in][r]{#1}}
     \def\centbox#1{\makebox[0in]{#1}}
     \def\topbox#1{\raisebox{-\baselineskip}[0in][0in]{#1}}
     \def\midbox#1{\raisebox{-0.5\baselineskip}[0in][0in]{#1}}

\vspace{3cm}

\title{ 
%\logo{
Digital Signal Processing
%}
%	\logo{Octave for Math Computing }
}
%\title{
%	\logo{Matrix Analysis through Octave}{\begin{center}\includegraphics[scale=.24]{tlc}\end{center}}{}{HAMDSP}
%}


% paper title
% can use linebreaks \\ within to get better formatting as desired
%\title{Matrix Analysis through Octave}
%
%
% author names and IEEE memberships
% note positions of commas and nonbreaking spaces ( ~ ) LaTeX will not break
% a structure at a ~ so this keeps an author's name from being broken across
% two lines.
% use \thanks{} to gain access to the first footnote area
% a separate \thanks must be used for each paragraph as LaTeX2e's \thanks
% was not built to handle multiple paragraphs
%

\author{ Mannem Charan AI21BTECH11019}
% note the % following the last \IEEEmembership and also \thanks - 
% these prevent an unwanted space from occurring between the last author name
% and the end of the author line. i.e., if you had this:
% 
% \author{....lastname \thanks{...} \thanks{...} }
%                     ^------------^------------^----Do not want these spaces!
%
% a space would be appended to the last name and could cause every name on that
% line to be shifted left slightly. This is one of those "LaTeX things". For
% instance, "\textbf{A} \textbf{B}" will typeset as "A B" not "AB". To get
% "AB" then you have to do: "\textbf{A}\textbf{B}"
% \thanks is no different in this regard, so shield the last } of each \thanks
% that ends a line with a % and do not let a space in before the next \thanks.
% Spaces after \IEEEmembership other than the last one are OK (and needed) as
% you are supposed to have spaces between the names. For what it is worth,
% this is a minor point as most people would not even notice if the said evil
% space somehow managed to creep in.



% The paper headers
%\markboth{Journal of \LaTeX\ Class Files,~Vol.~6, No.~1, January~2007}%
%{Shell \MakeLowercase{\textit{et al.}}: Bare Demo of IEEEtran.cls for Journals}
% The only time the second header will appear is for the odd numbered pages
% after the title page when using the twoside option.
% 
% *** Note that you probably will NOT want to include the author's ***
% *** name in the headers of peer review papers.                   ***
% You can use \ifCLASSOPTIONpeerreview for conditional compilation here if
% you desire.




% If you want to put a publisher's ID mark on the page you can do it like
% this:
%\IEEEpubid{0000--0000/00\$00.00~\copyright~2007 IEEE}
% Remember, if you use this you must call \IEEEpubidadjcol in the second
% column for its text to clear the IEEEpubid mark.



% make the title area
\maketitle

%\newpage

\tableofcontents

%\renewcommand{\thefigure}{\thesection.\theenumi}
%\renewcommand{\thetable}{\thesection.\theenumi}

\renewcommand{\thefigure}{\theenumi}
\renewcommand{\thetable}{\theenumi}

%\renewcommand{\theequation}{\thesection}


\bigskip

\begin{abstract}
This document provides the solution of Assignment 2.
\end{abstract}

\textbf{Question 2.12 :}Consider a system with input $x[n]$ and output $y[n]$ that satisfy the differential equation 
   \begin{align}
	   y[n] &= ny[n-1] + x[n]\label{def:diff}
   \end{align}
The system is causal and satisfies initial-rest conditions i.e., if $x[n] = 0$ for $n < n_{0}$, then $y[n] = 0$ for $n < n_{0}$.
\begin{enumerate}[label = (\alph*)]
	\item If $x[n] = \delta[n]$, determine $y[n]$ for all $n$.
        \item Is the system linear? Justify your answer.
        \item Is the system time invariant? Justify your answer.
\end{enumerate}

\solution Given that system is causal which means the output $y[n]$ only depends on present and past inputs $\brak{x[n]}$ but not the future inputs and also it satisfies initial-rest conditions.

\begin{enumerate}[label = (\alph*)]
	\item Given that $x[n] = \delta[n]$,
		\begin{align}
	      \implies x[n] &= \begin{cases}
			     1 &, n = 0 \\
			     0 &, \text{otherwise}
                         \end{cases}
                \end{align}
Now since system satisfies initial-rest conditions, 
              \begin{align}
		      y[n] &= 0 \text{for} n <0 \\
		      \because x[n] = 0 \text{for} n < 0 \nonumber
              \end{align}
	      Now put $n = 0$ for $\eqref{def:diff}$,
	      \begin{align}
		      y[0] &= 0+ x[0] \\
		           &= 1
	      \end{align}
	Now for $n > 0$,
	 \begin{align}
		 y[n] &= ny[n-1] + 0 \\
             \implies &= n\brak{n-1}y[n-2] \\
		      &= n\brak{n-1}\brak{n-2}\cdots y[0] \\
	     \implies &= n! \brak{\because y[0] = 1} 
          \end{align}
	So, overall the output $y[n]$ will be,
	 \begin{align}
		 y[n] &= \begin{cases}
			        0  &, n <0 \\
				n! &, n \geq 0
		         \end{cases}\\
			 \implies y[n] &= n!u[n] \label{result}
	  \end{align}
    \item To determine whether a system is linear, it should satisfy the following condition,
	     \begin{align}
	       T\cbrak{ax_{1}\sbrak{n} + bx_2\sbrak{n}} &= aT\cbrak{x_{1}\sbrak{n}} + bT\cbrak{x_{1}\sbrak{n}} \\
	                                                &= ay_{1}\sbrak{n} + by_{2}\sbrak{n}
             \end{align}
	     where $T$ is underlying transformation from input to output in the system. \\
	So here, take the input signal as superposition of two Dirac-delta functions,
	 \begin{align}
            x\sbrak{n} &= a\delta\sbrak{n} + b\delta\sbrak{n}
	 \end{align}
	Similarly we will calculate the output signal as above,
	 \begin{align}
	     y[n] &= 0 , n < 0 \\
	     y[0] &= x[0] \\
	          &= a + b 
	 \end{align}
	 And for $n > 0$,
	  \begin{align}
		  y[n] &= ny[n-1] \\
		       &= n!\brak{a+b}
	  \end{align}
	So for all $n$,
	  \begin{align}
		y[n] &= \brak{a+b}n!u\cbrak{n} \\
		\implies T\cbrak{a\delta\sbrak{n} + b\delta\sbrak{n}} &= a\brak{n!u[n]} + b\brak{n!u[n]} \\
								      &= aT\cbrak{\delta[n]} + bT\cbrak{\delta[n]} 
          \end{align}
	  $\therefore$ The given system is \textbf{linear}.
  \item To determine whether a system is Time-Invariant, the delay of $n_{0}$ in the input signal should result in same delay of $n_{0}$ in the output signal.\\
	Here take the input signal as,
	 \begin{align}
	    x[n] &= \delta\sbrak{n-n_{0}} 
	 \end{align}
     So using initial-rest conditions,
       \begin{align}
	       y[n] &= 0 \, \,\text{for} \, \, n < n_{0}
       \end{align}
  And using the $\eqref{def:diff}$,
      \begin{align}
	  y[n_{0}] &= 0 + x[n_0] \\
	           &= 1
      \end{align}
  And for $n > n_0$,
  \begin{align}
	  y[n] &= ny[n-1] + 0 \\
	       &= n\brak{n-1}\brak{n-2}\cdots \brak{n_0+1} \\
	       &= \frac{n!}{\brak{n_0}!}
  \end{align}
 So overall,
       \begin{align}
	       y[n] &= \brak{\frac{n!}{n_0!}}u[n-n_0]
       \end{align}
       Whereas using \eqref{result}, you will get,
        \begin{align}
		y'[n] &= (n-n_0)!u[n-n_0] \\
		      &\neq y[n]
	\end{align}
 $\therefore$ The system is not a \textbf{Time-Invariant} system.
\end{enumerate}


\end{document}
