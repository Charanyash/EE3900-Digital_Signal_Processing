\documentclass[journal,12pt,twocolumn]{IEEEtran}
%
\usepackage{setspace}
\usepackage{gensymb}
\usepackage{xcolor}
\usepackage{caption}
%\usepackage{subcaption}
%\doublespacing
\singlespacing
\usepackage{polynom}
%\usepackage{graphicx}
%\usepackage{amssymb}
%\usepackage{relsize}
\usepackage[cmex10]{amsmath}
\usepackage{mathtools}
%\usepackage{amsthm}
%\interdisplaylinepenalty=2500
%\savesymbol{iint}
%\usepackage{txfonts}
%\restoresymbol{TXF}{iint}
%\usepackage{wasysym}
\usepackage{hyperref}
\usepackage{amsthm}
\usepackage{mathrsfs}
\usepackage{txfonts}
\usepackage{stfloats}
\usepackage{cite}
\usepackage{cases}
\usepackage{subfig}
%\usepackage{xtab}
\usepackage{longtable}
\usepackage{multirow}
%\usepackage{algorithm}
%\usepackage{algpseudocode}
%\usepackage{enumerate}
\usepackage{enumitem}
\usepackage{mathtools}
%\usepackage{iithtlc}
%\usepackage[framemethod=tikz]{mdframed}
\usepackage{listings}


%\usepackage{stmaryrd}


%\usepackage{wasysym}
%\newcounter{MYtempeqncnt}
\DeclareMathOperator*{\Res}{Res}
%\renewcommand{\baselinestretch}{2}

%\renewcommand{\labelenumi}{\textbf{\theenumi}}
%\renewcommand{\theenumi}{P.\arabic{enumi}}

% correct bad hyphenation here
\hyphenation{op-tical net-works semi-conduc-tor}

\lstset{
language=Python,
frame=single, 
breaklines=true,
columns=fullflexible
}



\begin{document}
%

\theoremstyle{definition}
\newtheorem{theorem}{Theorem}[section]
\newtheorem{problem}{Problem}
\newtheorem{proposition}{Proposition}[section]
\newtheorem{lemma}{Lemma}[section]
\newtheorem{corollary}[theorem]{Corollary}
\newtheorem{example}{Example}[section]
\newtheorem{definition}{Definition}[section]
%\newtheorem{algorithm}{Algorithm}[section]
%\newtheorem{cor}{Corollary}
\newcommand{\BEQA}{\begin{eqnarray}}
\newcommand{\EEQA}{\end{eqnarray}}
\newcommand{\define}{\stackrel{\triangle}{=}}

\bibliographystyle{IEEEtran}
%\bibliographystyle{ieeetr}

\providecommand{\nCr}[2]{\,^{#1}C_{#2}} % nCr
\providecommand{\nPr}[2]{\,^{#1}P_{#2}} % nPr
\providecommand{\mbf}{\mathbf}
\providecommand{\pr}[1]{\ensuremath{\Pr\left(#1\right)}}
\providecommand{\qfunc}[1]{\ensuremath{Q\left(#1\right)}}
\providecommand{\sbrak}[1]{\ensuremath{{}\left[#1\right]}}
\providecommand{\lsbrak}[1]{\ensuremath{{}\left[#1\right.}}
\providecommand{\rsbrak}[1]{\ensuremath{{}\left.#1\right]}}
\providecommand{\brak}[1]{\ensuremath{\left(#1\right)}}
\providecommand{\lbrak}[1]{\ensuremath{\left(#1\right.}}
\providecommand{\rbrak}[1]{\ensuremath{\left.#1\right)}}
\providecommand{\cbrak}[1]{\ensuremath{\left\{#1\right\}}}
\providecommand{\lcbrak}[1]{\ensuremath{\left\{#1\right.}}
\providecommand{\rcbrak}[1]{\ensuremath{\left.#1\right\}}}
\theoremstyle{remark}
\newcommand\Mydiv[2]{%
$\strut#1$\kern.25em\smash{\raise.3ex\hbox{$\big)$}}$\mkern-8mu
        \overline{\enspace\strut#2}$}
\newtheorem{rem}{Remark}
\newcommand{\sgn}{\mathop{\mathrm{sgn}}}
\providecommand{\abs}[1]{\left\vert#1\right\vert}
\providecommand{\res}[1]{\Res\displaylimits_{#1}} 
\providecommand{\norm}[1]{\lVert#1\rVert}
\providecommand{\mtx}[1]{\mathbf{#1}}
\providecommand{\mean}[1]{E\left[ #1 \right]}
\providecommand{\fourier}{\overset{\mathcal{F}}{ \rightleftharpoons}}
\providecommand{\ztrans}{\overset{\mathcal{Z}}{ \rightleftharpoons}}
\newcommand{\myvec}[1]{\ensuremath{\begin{pmatrix}#1\end{pmatrix}}}
\newcommand{\mydet}[1]{\ensuremath{\begin{vmatrix}#1\end{vmatrix}}}
%\providecommand{\hilbert}{\overset{\mathcal{H}}{ \rightleftharpoons}}
\providecommand{\system}{\overset{\mathcal{H}}{ \longleftrightarrow}}
	%\newcommand{\solution}[2]{\textbf{Solution:}{#1}}
\newcommand{\solution}{\noindent \textbf{Solution: }}
\providecommand{\dec}[2]{\ensuremath{\overset{#1}{\underset{#2}{\gtrless}}}}

%\numberwithin{equation}{subsection}
%\numberwithin{problem}{subsection}
%\numberwithin{definition}{subsection}
\makeatletter
\@addtoreset{figure}{problem}
\makeatother

\let\StandardTheFigure\thefigure
%\renewcommand{\thefigure}{\theproblem.\arabic{figure}}
\renewcommand{\thefigure}{\theproblem}


%\numberwithin{figure}{subsection}

\def\putbox#1#2#3{\makebox[0in][l]{\makebox[#1][l]{}\raisebox{\baselineskip}[0in][0in]{\raisebox{#2}[0in][0in]{#3}}}}
     \def\rightbox#1{\makebox[0in][r]{#1}}
     \def\centbox#1{\makebox[0in]{#1}}
     \def\topbox#1{\raisebox{-\baselineskip}[0in][0in]{#1}}
     \def\midbox#1{\raisebox{-0.5\baselineskip}[0in][0in]{#1}}

\vspace{3cm}

\title{ 
%\logo{
Digital Signal Processing
%}
%	\logo{Octave for Math Computing }
}
%\title{
%	\logo{Matrix Analysis through Octave}{\begin{center}\includegraphics[scale=.24]{tlc}\end{center}}{}{HAMDSP}
%}


% paper title
% can use linebreaks \\ within to get better formatting as desired
%\title{Matrix Analysis through Octave}
%
%
% author names and IEEE memberships
% note positions of commas and nonbreaking spaces ( ~ ) LaTeX will not break
% a structure at a ~ so this keeps an author's name from being broken across
% two lines.
% use \thanks{} to gain access to the first footnote area
% a separate \thanks must be used for each paragraph as LaTeX2e's \thanks
% was not built to handle multiple paragraphs
%

\author{ Mannem Charan AI21BTECH11019}
% note the % following the last \IEEEmembership and also \thanks - 
% these prevent an unwanted space from occurring between the last author name
% and the end of the author line. i.e., if you had this:
% 
% \author{....lastname \thanks{...} \thanks{...} }
%                     ^------------^------------^----Do not want these spaces!
%
% a space would be appended to the last name and could cause every name on that
% line to be shifted left slightly. This is one of those "LaTeX things". For
% instance, "\textbf{A} \textbf{B}" will typeset as "A B" not "AB". To get
% "AB" then you have to do: "\textbf{A}\textbf{B}"
% \thanks is no different in this regard, so shield the last } of each \thanks
% that ends a line with a % and do not let a space in before the next \thanks.
% Spaces after \IEEEmembership other than the last one are OK (and needed) as
% you are supposed to have spaces between the names. For what it is worth,
% this is a minor point as most people would not even notice if the said evil
% space somehow managed to creep in.



% The paper headers
%\markboth{Journal of \LaTeX\ Class Files,~Vol.~6, No.~1, January~2007}%
%{Shell \MakeLowercase{\textit{et al.}}: Bare Demo of IEEEtran.cls for Journals}
% The only time the second header will appear is for the odd numbered pages
% after the title page when using the twoside option.
% 
% *** Note that you probably will NOT want to include the author's ***
% *** name in the headers of peer review papers.                   ***
% You can use \ifCLASSOPTIONpeerreview for conditional compilation here if
% you desire.




% If you want to put a publisher's ID mark on the page you can do it like
% this:
%\IEEEpubid{0000--0000/00\$00.00~\copyright~2007 IEEE}
% Remember, if you use this you must call \IEEEpubidadjcol in the second
% column for its text to clear the IEEEpubid mark.



% make the title area
\maketitle

%\newpage

\tableofcontents

%\renewcommand{\thefigure}{\thesection.\theenumi}
%\renewcommand{\thetable}{\thesection.\theenumi}

\renewcommand{\thefigure}{\theenumi}
\renewcommand{\thetable}{\theenumi}

%\renewcommand{\theequation}{\thesection}


\bigskip

\begin{abstract}
This document provides the solution of Assignment 1.
\end{abstract}
\textbf{Question 3.6.e :} Determine the inverse z-transform using both the methods- partial fraction expansion and power series expansion.In addition, indicate whether the Fourier transform exists or not.
  \begin{align}
    X(z) &= \frac{1-az^{-1}}{z^{-1} - a} , \, \, \abs{z} > \abs{\frac{1}{a}}
  \end{align}
 \solution Given z-transform of a signal $x\brak{n}$ as,
   \begin{align}
     X(z) &= \frac{1-az^{-1}}{z^{-1} - a} , \, \, \abs{z} > \abs{\frac{1}{a}}\label{eq:z-transform}
   \end{align}
  Now to find inverse z-transform of eq $\eqref{eq:z-transform}$,we will use following two methods,
   \begin{enumerate}
    \item Using partial fractions
    \item Using power series expansion
   \end{enumerate}
   \textbf{Using partial fractions:} \\
     \begin{align}
      X(z) &= \frac{1-az^{-1}}{z^{-1} - a} , \, \, \abs{z} > \abs{\frac{1}{a}} \\
         &= \frac{z^1\brak{z-a}}{z^1\brak{1-az^1}} 
     \end{align}
    From the region of convergence i.e., ROC $x\brak{n}$ is a right-sided sequence and since $M = N = 1$ and pole is first order,we can write $X\brak{z}$ as,
     \begin{align}
      X(z) &= b_{0} + \frac{A_{1}}{1-rz^{-1}} \label{eq:2}
     \end{align}
    Now 
     \begin{align}
       X(z) &= \frac{1 - az^{-1}}{-a\brak{1 - \frac{z^{-1}}{a}}}\label{eq:3}
     \end{align}   
    Comparing eqs $\eqref{eq:2}$ and $\eqref{eq:3}$, we will get $r = \frac{1}{a}$, and 
     \begin{align}
       b_{0}r &= -1 \\
       \implies b_{0} &= -a \, \, \,  \text{and}\\ 
       b_{0} + A_{1} &= \frac{-1}{a} \\
       \implies A_{1} &= \frac{-1}{a} + a
     \end{align}
    So $X(z)$ will be,
     \begin{align}
       X(z) &=  -a - \frac{\frac{1}{a} - a}{1 - \frac{z^{-1}}{a}}
     \end{align}
    Since ROC $\abs{z} > \abs{\frac{1}{a}}$,
     \begin{align}
         -a \ztrans -a\delta\brak{n}\\
         \frac{\frac{1}{a} - a}{1 - \frac{z^{-1}}{a}} \ztrans \brak{\frac{1}{a} - a}\brak{\frac{1}{a}}^nu\brak{n}\\
     \end{align}
     From the linearity of z-transform,
     \begin{align}
       x(n) &= -a\delta\brak{n} -\brak{1-a^2}a^{-\brak{n+1}}u\brak{n}
     \end{align}
    \textbf{Using power series expansion} Here we will try to find the power series itself which results in z-transform.And compare it with the below expression to get $x\brak{n}$,
      \begin{align}
        X\brak{z} &= \sum_{n=-\infty}^{\infty}x(n)z^{-n}\label{eq:4}
      \end{align}
    We will do that using long division method,\\
    $\begin{array}{llllll}
      & -\frac{1}{a} & -\frac{1}{a}(\frac{1}{a}-a)z^{-1}&-\frac{1}{a^2}\brak{\frac{1}{a}-a}z^{-2}&.\,. \\\cline{2-5}
      z^{-1} - a| & 1& -az^{-1}  \\
      & 1 & -\frac{z^{-1}}{a}  \\\cline{2-5}
      & & \frac{z^{-1}}{a} -az^{-1}& \\
      & & \brak{\frac{1}{a}-a}z^{-1} & -\frac{1}{a}\brak{\frac{1}{a}-a}z^{-2} & \\\cline{3-5}
      & & & \frac{1}{a}\brak{\frac{1}{a} - a}z^{-2} \\
      & & & .\,.
      \end{array}$
     The power series will be
      \begin{align}
        X\brak{z} &= -\frac{1}{a}+ -\brak{\frac{1}{a} - a}\sum_{n=1}^{\infty}\brak{\frac{1}{a}}^{n}z^{-n}
      \end{align}         
      Comparing with $\eqref{eq:4}$,
       \begin{align}
         x\brak{n} &= -a\delta\brak{n} - \brak{\frac{1}{a} - a}\brak{\frac{1}{a}}^nu\brak{n} \\
                   &=  -a\delta\brak{n} -\brak{1-a^2}a^{-\brak{n+1}}u\brak{n}
       \end{align}
      And for $x\brak{n}$, fourier transform exists when ROC contains the unit circle $\brak{\abs{z} = 1}$,this will happen when 
      \begin{align}
         \abs{\frac{1}{a}} < 1
         \implies \abs{a} > 1
      \end{align}
      $\therefore$ For $\abs{a} > 1$ , given signal $x\brak{z}$ has Fourier transform.  
      \end{document}